% problem 1
\section{Optimization}
Explain each of the following optimizations and show how they can be used to improve the three-address flow graph.
Start with the original flow graph for each case.

\emph{Note: I use ``Bn.m'' to refer to line m in block n of the original flow graph.}
\subsection{Common subexpression elimination}
% CU lecture 23
% DB pg 588
From \textsc{the Dragon Book}, page 588:
\begin{quote}
``An occurrence of an expression \emph{E} is called a common subexpression if \emph{E} was previously computed and the values of the variables in \emph{E} have not changed since the previous computation.''
\end{quote}

You'd think \textit{B3.6} contains a common subexpression, since \texttt{i + t} is also computed at \textit{B3.2}.
However \textit{B3.2} changes the value of \texttt{i}, so while the expression in \textit{B3.6} was previously computed, the values of the variables have changed and therefore it is not a common subexpression.

\subsection{Copy propagation}
% CU lecture 23
% DB pg 590
``After an assignment, \texttt{x = y}, replace references to \texttt{x} with \texttt{y} until \texttt{x} is assigned something else.''
In this quote, taken from the 23rd CU lecture slides, \texttt{y} probably denotes a variable rather than an expression.

This optimization would change \textit{B3.5} to \texttt{d = i + 2}.

You'd think we should also change \textit{B3.2} to \texttt{i = 1 + t}, because of the assignment in \textit{B1.1}, but doing so would gunk up the code because control can enter \textit{B3} through more than one path ($\mathit{B1} \rightarrow \mathit{B2} \rightarrow \mathit{B3}$ and $\mathit{B3} \rightarrow \mathit{B2} \rightarrow \mathit{B3}$).

\subsection{Code motion}
% DB pg 592
This is when we move the evaluation of a statement that doesn't change over time out of the body of the loop so that it's only computed once rather than once per iteration of the loop.

This optimization would move \textit{B3.1} and \textit{B3.4} to \textit{B1}.

\subsection{Dead code elimination}
% DB pg 591

Dead code elimination is when we remove statements whose effects are never observed in the code.
E.g. if we've got two different assignments to the same variable without the variable ever being referenced in-between the two assignments, then we can remove the first assignment as its effect is never observed.

This would remove \textit{B3.4}. 

% problem 2
\newpage
\section{[More] Optimization}
Consider the following C-code:
\begin{lstlisting}[language=C, tabsize=4, basicstyle=\ttfamily\small]
for (int i =0; i < n; i++) {
	sum = 4 * i;
	for (int j = 0; j < m; j = j + i) {
		a = a + b * 2;
	}
}
\end{lstlisting}

\subsection{In the context of optimization, what is an induction variable, and what is reduction in strength?}
% DB pg 592
\paragraph{Induction variable}
A variable $x$ is said to be an ``induction variable'' if there is a constant $c \in \mathbb{R}$ such that each time $x$ is assigned, its value increases by $c$.

\texttt{i}, as in the textbook example and programmer's favourite loop-counter-variable, is an induction variable.
Who knows how that came to be?
Is it because of those Fortran-people being math-people and mathematicians use $i$ in their sums and matrix indexes?
Does it stand for ``\textbf{i}ndex''?
Oh convention, you are surely the mother of some confusion.

\paragraph{Reduction in strength}
In the context of optimization, it is the replacement of ``expensive'' operations (such as multiplication and division) by a cheaper one (such as addition or subtraction).

\subsection{Convert the code above to a three-address flow graph.}
% I could use flow (http://ctan.mackichan.com/support/flow/flowdoc.pdf) for this
% or I could man up and do it with TikZ, but then I'd have to learn TikZ.
%	http://www.texample.net/tikz/examples/simple-flow-chart/
% Ooorr apparently I could use graphviz. I already sort-of know graphviz, so hmm.
% man, so many choices.
% fuck it let's just use TikZ
%	http://www.texample.net/media/pgf/builds/pgfmanualCVS2012-11-04.pdf
\begin{figure}[H]
\centering
	% hurp a derp a tikz flow graph

\tikzstyle{block} = [rectangle, draw, font=\ttfamily\small]
\tikzstyle{arrow} = [draw, -latex']
\tikzstyle{label} = [font=\fontshape{it}\selectfont] % labels are in italics

\begin{tikzpicture}[node distance = 2cm, auto]
	% code blocks
	\node[block] (B1) 				{i = 0};
	\node[block, below of=B1] (B2) 	{if i < n goto B3};
	%	edge[arrow, bend right=45] (B1.east);
	\node[block, below of=B2] (B3)	{\makecell[l]{c = 4 * i\\j = 0}};
	\node[block, below of=B3] (B4)	{\makecell[l]{if j < m goto B5}};
	\node[block, below of=B4] (B5)	{\makecell[l]{d = b * 2 \\a = a + d \\j = j + i}};
	\node[block, below of=B5] (B6)	{i = i + 1};
	\node[block, below of=B6] (B7)	{ };
	% labels
	\node[label, left of=B1] 		{B1};
	\node[label, left of=B2] 		{B2};
	\node[label, left of=B3]		{B3};
	\node[label, left of=B4]		{B4};
	\node[label, left of=B5]		{B5};
	\node[label, left of=B6]		{B6};
	\node[label, left of=B7]		{B7};
	% edges
	\path	(B1)		edge	[arrow, bend right=0]	node	{}	(B2);
	\path	(B2)		edge	[arrow]					node	{}	(B3);
	\path	(B3)		edge	[arrow]					node	{}	(B4);
	\path	(B4)		edge	[arrow]					node	{}	(B5);
	\path	(B6.east)	edge	[arrow, bend right=45]	node	{}	(B2);
	\path	(B2.east)	edge	[arrow, bend left=55]	node	{}	(B7.east);
	\path	(B4.east)	edge	[arrow, bend left=30]	node	{}	(B6);
	\path	(B5.west)	edge	[arrow, bend left=50]	node	{}	(B4.west);
\end{tikzpicture}

\caption{A three-address flow graph for the code in Problem 2.}
\label{fig:2-b}
\end{figure}

\subsection{Optimize the flow graph by performing strength reduction on induction variables, and removing unnecessary induction variables.}
% strength reduction on the induction variables
% remove unnecessary induction variables

% sum = 4*i ==> sum = sum + 4
% j is only used for the comparison in block 3
% i is used in the computation of j and in the comparison in block 2
Ok so for the $n$th iteration of the inner loop, the value of \texttt{j} is $j_n$ where $j_n = j_{n-1} + i$ and $j_1 = 0$.
If we work upwards from $j_1$, we see that $j_n = (n-1)*i$.
I'm not sure if it helps, though.
All I managed to do with that was replace \texttt{j = i + j} with \texttt{j = i * n} where \texttt{n} is a new induction variable that increases by 1 for each iteration of the inner loop.
But not only does that introduce a new induction variable, it's also the opposite of strength reduction.
So I didn't do that.

I did do code movement and pulled \texttt{d = b * 2} out of the loop body and also reduced it to \texttt{d = b << 1}.

Also I replaced the multiplication in the assignment of \texttt{sum} with addition.
% d = b << 1
% i also did code motion

\begin{figure}[H]
\centering
	% hurp a derp a tikz flow graph

\tikzstyle{block} = [rectangle, draw, font=\ttfamily\small]
\tikzstyle{arrow} = [draw, -latex']
\tikzstyle{label} = [font=\fontshape{it}\selectfont] % labels are in italics

\begin{tikzpicture}[node distance = 2cm, auto]
	% code blocks
	\node[block] (B1) 				{\makecell[l]{i = 0 \\c = - 4 \\d = b << 1}};
	\node[block, below of=B1] (B2) 	{if i < n goto B3};
	%	edge[arrow, bend right=45] (B1.east);
	\node[block, below of=B2] (B3)	{\makecell[l]{c = c + 4 \\j = 0}};
	\node[block, below of=B3] (B4)	{\makecell[l]{if j < m goto B5}};
	\node[block, below of=B4] (B5)	{\makecell[l]{a = a + d \\j = j + i}};
	\node[block, below of=B5] (B6)	{i = i + 1};
	\node[block, below of=B6] (B7)	{ };
	% labels
	\node[label, left of=B1] 		{B1};
	\node[label, left of=B2] 		{B2};
	\node[label, left of=B3]		{B3};
	\node[label, left of=B4]		{B4};
	\node[label, left of=B5]		{B5};
	\node[label, left of=B6]		{B6};
	\node[label, left of=B7]		{B7};
	% edges
	\path	(B1)		edge	[arrow, bend right=0]	node	{}	(B2);
	\path	(B2)		edge	[arrow]					node	{}	(B3);
	\path	(B3)		edge	[arrow]					node	{}	(B4);
	\path	(B4)		edge	[arrow]					node	{}	(B5);
	\path	(B6.east)	edge	[arrow, bend right=45]	node	{}	(B2);
	\path	(B2.east)	edge	[arrow, bend left=55]	node	{}	(B7.east);
	\path	(B4.east)	edge	[arrow, bend left=30]	node	{}	(B6);
	\path	(B5.west)	edge	[arrow, bend left=50]	node	{}	(B4.west);
\end{tikzpicture}

\caption{An optimized version of the three-address flow graph in Figure~\ref{fig:2-b}.}
\label{fig:2-c}
\end{figure}

% problem 3
\newpage
\section{Data-flow analysis}
\subsection{In the context of reaching definition analysis, what does it mean for a definition to reach a point in the code?}
% I guess it's time to read 9.2 of DB
% pg 598: a variable may be defined by one of [some number of points in the code]. The definitions that may reach a program point along some path are known as reaching definitions.
% So if we're at some point X in the code, then all the definitions at points that might be included in the path to X are reaching definitions? ok I think I got that
% ooh so if the set of reaching definitions for variable x only has one member at a point Y then we can replace x by that definition at point Y (constant folding!)
If a definition for some variable $x$ reaches a point $p$ in the code, then there exists a path from the definition to $p$ on which there are no other definitions of $x$.
The definitions need not be constants ($x$ could be assigned another variable, for instance).
However if there is only one reaching definition for $x$ at a point right before $x$ is referenced then it's possible to replace $x$ with that definition.
If the definition is a constant it's called \emph{constant folding}.
If it's a variable it is called \emph{copy propagation}.

\subsection{Perform reaching definitions data-flow analysis by computing the \emph{gen, kill}, IN and OUT sets for each block of the flow graph above.}
% IN[s] = data-flow values before statement s
% OUT[s] = data-flow values after statement s
% data-flow value = abstract representation of the set of all possible program states that can be observed for a given point/statement

% transfer function of definition d: u = v + w (i.e. d is a definition of u)
% f_d(x) = gen_d \union (x - kill_d)
%	gen_d = {d}, kill_d = all other definitions of u in the program
% the fuck is x, though? Is it a point?

%anyway, gen(Block) is the set of all definitions that reach the end of the block, yeah?
% Definitions of gen and kill for basic blocks on pg 605 of DB
% kill(Block) is the union of all the definitions killed by the individual statements
% gen(Block) contains all the definitions inside the block that are "visible" immediately after the block -- we refer to them as 'downwards exposed'.
%	a definition is downward exposed if it is not overwritten by another definition below it

% so I just need to figure out what IN[Block] and OUT[Block] is
%pg 606:
% OUT[B] = gen(B) U (IN[B] - kill(B))
% IN[B]  = U_P OUT[P]
% the last one means that IN[B] is equal to the union of the OUT sets of blocks P that are direct predecessors of B
% "A - B" is the set of elements that are in A but not in B
% A U \emptyset = A

% I just learned how to do this!
\newcommand{\IN}[1]{\text{IN[#1]}}
\newcommand{\OUT}[1]{\text{OUT[#1]}}

So $gen(\text{B})$ is the set of all definitions inside block B that reach the end of the block and $kill(\text{B})$ is the union of all the definitions killed by the individual statements in the block.
Note that according to Figure 9.13 on page 604 it would seem that the domain of $kill(\text{B}$ is all definitions in the program (or CFG, whichever).

IN[B] and OUT[B] are defined as
$$\OUT{B} = gen(B) \cup (\IN{B} - kill(B))$$
$$\IN{B} = \bigcup_{P}\OUT{P}$$
Where P is a predecessor block of B and I'm guessing the big cup thing is supposed to be like a sum except instead of addition it performs union?
It's from page 605 of \textsc{BOOK OF DRAGONS}.

Also
$$\OUT{ENTRY} = \emptyset$$

An algorithm for computing the IN and OUT sets for the blocks of a CFG can be found in Figure 9.14 on page 607 of \textsc{You know which book}.
That's the one I used to figure out the values seen in Table~\ref{tab:3-b-1}~through~\ref{tab:3-b-3}.
Well actually I stopped once I saw that \OUT{B1} was unchanged between Step 2 and 3, because the algorithm does say we should stop when there are no more changes to any OUT and since the OUT set of a block depends on its IN set which depends on the OUT set of its immediate predecessors I figured we wouldn't be seeing any more changes due to how the graph looks.
Yeah that's a rather convoluted explanation but I'm sure you know how it's supposed to work.

% man I am pretty lazy
\renewcommand\thesubsubsection{Step \arabic{subsubsection}}
\setcounter{subsubsection}{-1} % counters are apparently inserted as ++counter
% so if you want the first occurrence of it to be n, you have to set it to n-1
\subsubsection{\emph{gen} and \emph{kill} sets}

\newcommand{\blockn}{B1}
$gen(\text{B1}) = \{d_1,~d_2,~d_3\}$

$kill(\text{B1}) = \{d_5, ~d_6, ~d_8, ~d_{10}, ~d_{11}, ~d_{12}\}$

$gen(\text{B2}) = \{d_4,~d_5\}$

$kill(\text{B2}) = \{d_1,~ d_{10}\}$

$gen(\text{B3}) = \{d_6,~ d_7\}$

$kill(\text{B3}) = \{d_2,~ d_8\}$

$gen(\text{B4}) = \{d_8,~ d_9\}$

$kill(\text{B4}) = \{d_2,~ d_6\}$

$gen(\text{B5}) = \{d_{10},~ d_{12}\}$

$kill(\text{B5}) = \{d_1,~ d_3,~ d_5,~ d_{10},~ d_{11},~ d_{12}\}$

\subsubsection{First iteration}
% whoo-hoo! this is like using variables or whatever and generating the text!
% thug lyfe mofo
\newcommand{\INBone}	{$\emptyset$}
\newcommand{\OUTBone}	{$\{d_{1},~ d_2,~ d_3\}$}
\newcommand{\INBtwo}	{\OUTBone}
\newcommand{\OUTBtwo}	{$\{d_2,~ d_3,~ d_4,~ d_5\}$}
\newcommand{\INBthree}	{\OUTBtwo}
\newcommand{\OUTBthree}	{$\{d_3,~ d_4,~ d_5,~ d_6,~ d_7\}$}
\newcommand{\INBfour}	{\OUTBtwo}
\newcommand{\OUTBfour}	{$\{d_3,~ d_4,~ d_5,~ d_8,~ d_9\}$}
\newcommand{\INBfive}	{$\{d_3,~ d_4,~ d_5,~ d_6,~ d_7,~ d_8,~ d_9\}$}
\newcommand{\OUTBfive}	{$\{d_1,~ d_4,~ d_6,~ d_7,~ d_8,~ d_9,~ d_{10},~ d_{12}\}$}

\begin{table}[H]
\centering
\begin{tabular}{lcc}
	\toprule
	\textsc{Block}	& \textsc{IN}	& \textsc{OUT} \\
	\midrule
	B1	& \INBone	& \OUTBone	\\
	B2	& \INBtwo	& \OUTBtwo	\\
	B3	& \INBthree	& \OUTBthree\\
	B4	& \INBfour	& \OUTBfour	\\
	B5	& \INBfive	& \OUTBfive	\\
\end{tabular}
\label{tab:3-b-1}
\caption{Status of the IN and OUT sets at the end of step \arabic{subsubsection}.}
\end{table}


\subsubsection{Second iteration}
% can't use the values from the last step because LaTeX sees the re-definition of the commands
% before it sees their usage and apparently does lazy evaluation so the command inside the command isn't executed before the command is used, not when it is defined
% which, when you think about it, shouldn't be too surprising
% I guess I just wanted the compiler to know what I was thinking IS THAT TOO MUCH TO ASK
% HUH
\renewcommand{\INBone}		{$\{d_1,~ d_4,~ d_6,~ d_7,~ d_8,~ d_9,~ d_{10},~ d_{12}\}$}
\renewcommand{\OUTBone}		{$\{d_1,~ d_2,~ d_3,~ d_4,~ d_7,~ d_9\}$}
\renewcommand{\INBtwo}		{\OUTBone}
\renewcommand{\OUTBtwo}		{$\{d_2,~ d_3,~ d_4,~ d_5,~ d_7,~ d_9\}$}
\renewcommand{\INBthree}	{\OUTBtwo}
\renewcommand{\OUTBthree}	{$\{d_3,~ d_4,~ d_5,~ d_6,~ d_7,~ d_9\}$}
\renewcommand{\INBfour}		{\OUTBtwo}
\renewcommand{\OUTBfour}	{$\{d_3,~ d_4,~ d_5,~ d_7,~ d_8,~ d_9\}$}
\renewcommand{\INBfive}		{$\{d_3,~ d_4,~ d_5,~ d_6,~ d_7,~ d_8,~ d_9\}$}
\renewcommand{\OUTBfive}	{$\{d_4,~ d_6,~ d_7,~ d_8,~ d_9,~ d_{10},~ d_{12}\}$}

\begin{table}[H]
\centering
\begin{tabular}{lcc}
	\toprule
	\textsc{Block}	& \textsc{IN}	& \textsc{OUT} \\
	\midrule
	B1	& \INBone	& \OUTBone	\\
	B2	& \INBtwo	& \OUTBtwo	\\
	B3	& \INBthree	& \OUTBthree\\
	B4	& \INBfour	& \OUTBfour	\\
	B5	& \INBfive	& \OUTBfive	\\
\end{tabular}
\label{tab:3-b-1}
\caption{Status of the IN and OUT sets at the end of step \arabic{subsubsection}.}
\end{table}


\subsubsection{Third iteration}
\renewcommand{\INBone}		{$\{d_4,~ d_6,~ d_7,~ d_8,~ d_9,~ d_{10},~ d_{12}\}$}
\renewcommand{\OUTBone}		{$\{d_1,~ d_2,~ d_3,~ d_4,~ d_7,~ d_9\}$}

\begin{table}[H]
\centering
\begin{tabular}{lcc}
	\toprule
	\textsc{Block}	& \textsc{IN}	& \textsc{OUT} \\
	\midrule
	B1	& \INBone	& \OUTBone	\\
	B2	& \INBtwo	& \OUTBtwo	\\
	B3	& \INBthree	& \OUTBthree\\
	B4	& \INBfour	& \OUTBfour	\\
	B5	& \INBfive	& \OUTBfive	\\
\end{tabular}
\label{tab:3-b-1}
\caption{Status of the IN and OUT sets at the end of step \arabic{subsubsection}.}
\end{table}



\subsection{Explain how a compiler can use the results to determine that the variable \texttt{b} must be a constant at the start of the exit node.}
Oh, I forgot to compute \IN{EXIT} and \OUT{EXIT}.
Well I guess \OUT{EXIT} is either the empty set or equal to \IN{EXIT} depending on the program\footnote{for instance, if the CFG we're looking at is the entirety of the program then \OUT{EXIT} is the empty set unless some other program snatches the memory of this program and uses the same variables but if it's just a function or something inside a program then it might not be.}, but luckily I don't have to deal with that right now because the question asks about the value of a variable at the start of the EXIT node. \emph{Phew}.

\IN{EXIT} is computed the same way as any other IN set which, in our case, means it's equal to \OUT{B5}.
Looking at \OUT{B5} in Table~\ref{tab:3-b-3} reveals that it contains two definitions for the variable \texttt{b}: $d_6$ and $d_8$.
That means \texttt{b} could have either of these definitions as we reach the EXIT node.
But hey, \emph{wait a minute}, both $d_6$ and $d_8$ define \texttt{b} as a constant!
So that's how we know that the variable \texttt{b} must be a constant at the start of the exit node.
We don't know exactly \emph{which} constant it'll be, but it's either 5 or 4 which, y'know, is a smaller set than $\mathbb{Z}_{0}^{2^{32}-1}$ (or whatever, I don't know the type of \texttt{b}).
