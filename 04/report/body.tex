
\section{Problem 1, Symbol Tables}
\subsubsection{a) What is a symbol table, and why is it needed?}
yes
\subsubsection{b) What kind of information is typically stored in a symbol table?}
nope
\subsubsection{c) Mention and briefly discuss the advantages and disadvantages of three different data structures that can be used to implement symbol tables.}
ok


\newpage
\section{Problem 2, Symbol tables and blocks}
Consider the following code in a version of [VSL] that includes the keyword \texttt{BLOCK} to start a new scope:
\begin{figure}[H]
\begin{verbatim}
VOID FUNC main() START
    int a;
    FLOAT b;
    ...
    BLOCK
        bool b;
        ... // Position 1
    END

    BLOCK
        int b;
        FLOAT c;
        ...
        BLOCK
            BOOL a;
            INT c;
            ... // Position 2
        END
    END
END
\end{verbatim}
\end{figure}

\subsection{a) Show the contents of the symbol tables at position 1 and 2 assuming an implementation using a stack of symbol tables.}

\subsection{b) Show the cnotent of the symbol tables at position 1 and 2 assuming an implementation using a single symbol table.}

\subsection{c) What are the advantages and disadvantages of these approaches?}

\newpage
\section{Problem 3, Type checking}
\subsection{a) What is the difference between type synthesis and type inference?}

\subsection{b) Some languages, e.g. FORTRAN, have native support for complex numbers. Draw the widening conversions/hierarchy for the following set of types: \texttt{int}, \texttt{float}, \texttt{double}, \texttt{complex int}, \texttt{complex float}, \texttt{complex double}. State the assumptions you make.}


\newpage
\section{Problem 4, Types, SDDs}
Consider the following partially completed SDD for type expressions:
\begin{table}[H]
\begin{tabular}{ll}
	\textsc{Production} & \textsc{Semantic Rule} \\ \hline
	T ::= \textbf{int}	&						\\
	T ::= \textbf{float}& \\
	T ::= \textbf{bool} & \\
	T ::= \textbf{T[num]}& \\
	T ::= (L) 			& \\
	L ::= L,T			& \\
	L ::= T				& \\
\end{tabular}
\end{table}

Possible types are the basic types \textbf{int}, \textbf{float}, and \textbf{bool}, arrays of any kind of type, e.g. \textbf{int}[5] or \textbf{float}[5][4] and records (which are collections of one or more types, e.g. \textbf{int}, \textbf{float}, (\textbf{int, bool}[3])).

The purpose of the SDD is to compute the number of bytes that must be allocated to store a value of a given type.
\begin{itemize}
	\item \textbf{int} requires 4 bytes, \textbf{float} requires 8 bytes, \textbf{bool} requires 1 byte.
	\item An array requires the number of bytes a single element, multiplied by the length of the array. For instance, \textbf{int}[5] would require 20 bytes. Similarily, \textbf{int}[5][5] (where the type of the elements is \textbf{int}[5]) would require 100 bytes.
	\item A record requires the sum of bytes required by its members. That is, (\textbf{int, bool}[5]) would require 9 bytes.
\end{itemize}

\subsection{Complete the SDD, so that the \textit{size} attribute of T stores the size of the type.}

\subsection{Show annotated parse trees for each of these types:}
\begin{description}
	\item[i)] \textbf{int}[4][3][5]
	\item[ii)] (\textbf{int}, \textbf{float}, (\textbf{bool}, \textbf{int}[4]))

\end{description}

\newpage
\section{Problem 5, SDDs}
????
