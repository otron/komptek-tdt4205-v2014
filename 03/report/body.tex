\section{Problem 1, Parsing}
\subsection{a) What is the difference between a top-down parser and a bottom-up parser?}
% bottom-up parsing: pg 233
% top-down parsing: pg 61
A top-down parser begins with the grammar's starting symbol and tries to figure out what `moves' were taken to get from it to the input string.

Bottom-up parsers begin with the first terminal of the input and works backwards from it towards the grammar's starting symobl.


\subsection{b) What is the difference between a LL parser and a LR parser?}
% LL parser is a predictive parser, pg 64-8, 222-231
% LR parser, pg 53-252, 275-7
% A derivation of a string for a grammar is a sequence of grammar rule applications that transforms the start symbol into the string. A derivation proves that the string belongs to the grammar's language.

An LL parser parses the input from left to right and constructs a leftmost derivation of the sentence.
A leftmost derivation looks at the leftmost non-terminal first when deriving a string.

While LR parsers also parse the input from left to right, they construct a rightmost derivation of the sentence.
LR parsers are shift-reduce parsers.

Ohh, also while LR parsers \textit{can} be written by hand, they're typically not because it is a lot of work.

